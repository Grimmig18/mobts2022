\chapter{Imagefilm}
\section{Film, eine Einleitung}
\begin{quote}
    Ein Bild sagt mehr als tausend Worte
\end{quote}
\begin{flushright}
    Fred R. Barnard
\end{flushright}
Mit diesem Zitat lässt sich die Wichtigkeit des Films als kommunikationsmittel sehr gut beschreiben. So liefert ein Film nicht nur eine große Menge an Informationen durch Darstellung eines Bildes, sondern kann ein Film auch zeitliche Abläufe vermitteln. Oft haben Filme auch noch das weitere Medium des Tons, welches in sich nochmals mehr Informationen trägt. Ein Film kann deswegen als ein sehr \enquote{dichtes} Informationsträger angesehen werden, da in kurzer Zeit sehr viele Informationen gleichzeitig, teils durch verschiedene Kanäle wie Bild und Ton, übertragen werden. So ergibt auch der Einsatz dieses Mediums für die \ac{MOBTS} Konferenz in Mannheim Sinn.\\

Im Folgendem wird deshalb kurz die Geschichte des Films erläutert, um die Vorteile und die interessante Herkunft dieses Mediums darzustellen. Anschließend wird auf verschiedene Techniken des Filmens eingegangen, welche die Grundlagen der Filmproduktion darstellen. Danach wird auf die Umsetzung des Werbefilms für die \ac{MOBTS} Konferenz eingegangen, welcher anschließend bewertet wird. Zuletzt wird noch in einem Fazit ein kurzer Ausblick gegeben.
\section{Geschichte des Films}
Der Mensch suchte schon seit Ewigkeiten danach, seine Erfahrungen und Erlebnisse visuell darzustellen. So finden sich schon in den frühesten Zivilisationen verschiedenste Darstellungen der Umwelt. Noch bevor der Mensch Schrift erfand malte er auf Höhlenwände Bilder. Mit dem technischen Fortschritt entwickelte sich auch die Darstellungsformen, sodass Bilder auf Leinwände gemalt wurden, bis schließlich die Fotografie erfunden wurde. Bei der Fotografie wird ein lichtempfindlicher Film durch eine Linse mit Licht bestrahlt, was ein Abbild der Umwelt darstellt.\\
Um das Bild bewegt erscheinen zu lassen, erzeugt ein Film eine Illusion, indem mehrere aufeinanderfolgende Bilder schnell hintereinander dargestellt werden. Dies erscheint dann aber einer Bildrate von 24 Bildern pro Sekunde dem menschlichen Auge als flüssige Bewegung.\autocite{Schmidt.2013}\\
Auf der Suche nach dem allerersten \enquote{Film} gibt es drei wesentliche Kandidaten.\autocite{HeadsUp.} Der chronologisch erste Film wurde 1878 erstellt.\autocite{Muybridge.1878} Das Ziel dieses Films war es, herauszufinden, ob ein Pferd im Gallopp jemals mit allen vier Hufen gleichzeitig den Boden verlässt. Dafür Fotografierte Edward Muybridge ein Pferd in Palo Alto, Kalifornien im Lauf mehrfach mit verschiedenen Kameras, die entlang der Strecke aufgestellt waren. Diese verschiedenen Fotos wurden dann im Nachhinein schnell hintereinander gezeigt, um zu entdecken, dass ein Pferd mit allen vier Hufen gleichzeitig vom Boden abhebt. Hier war es jedoch nicht die Absicht von Muybridge, ein Bewegtbild zu erzeugen, weshalb dies oft nicht als erster Film angesehen wird.\\
Oft wird der Film Roundhay Garden Scene als der erste Film betitelt.\autocite{LePrince.1888} Der von Louis Amie Augustin Le Prince erstellte Film von 1888 geht zwar lediglich zwei Sekunden, stellt jedoch schon eine zusammenhängende Handlung dar und gilt deswegen als Film.\\
Etwas länger dagegen ist der Film \enquote{Arrival of a Train at A la Ciotat}, im original \enquote{L'Arrivée d'un Train A la Ciotat}, welcher 1895 von den Gebrüdern Lumiere erstellt wurde.\autocite{Lumiere.1895} Dieser 50-sekündige Film zeigt einen einfahrenden Zug, eine bekannte Alltagsszene. Zu diesem Film gibt es die unbestätigte Geschichte, welche besagt, dass die Zuschauer dieses Films von der Illusion in einem solchen Sinne getäuscht wurden, sodass sich Panik breit machte und einige Zuschauer aus Angst aus dem Theater flohen. Auch wenn diese Geschichte vielleicht nicht wahr ist, zeigt sie die Überzeugungskraft des Mediums und die Emotionen, die durch visuelle Einflüsse ausgelöst werden können.\\
Dieses Ziel blieb ein wichtiges Ziel in der Filmkunst und Regisseure versuchen bis heute, wirksam Emotionen in Zuschauern auszulösen.\\
So gab es einige wichtige Regisseure, welche die Filmindustrie zu dem machten, was sie heute ist.
\section{Techniken des Filmens}
\subsection{Framing}
\subsection{Goldener Schnitt}
\subsection{Skript und Storyboard}
\subsection{Schnitte und Übergänge}
\subsection{Voiceover}
\section{Imagefilm zur MOBTS}
\subsection{Ziel des Films}
\subsection{Skript des Films}
\subsection{Storyboard des Films}
\subsection{Durchführung und Produktion}
\section{Bewertung und Reflektion}
\section{Fazit und Ausblick}