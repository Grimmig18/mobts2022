\chapter{Konferenzplanung}
\section{Rahmenbedingungen}
Die MOBTS-Konferenz 2022 wird an der \ac{DHBW} in Mannheim stattfinden. \autocite[Vgl.][]{MOBTS.04.03.2021} Das folgende Konzept richtet sich an die Verantwortlichen der DHBW, die sich um die Umsetzung der Konferenz kümmern. Mithilfe des Konzepts soll die erfolgreiche Vorbereitung und Ausführung der Konferenz an der DHBW Mannheim unterstützt sowie wichtige Informationen für die Teilnehmer zusammengetragen werden. Durch die MOBTS-Konferenz sollen den Teilnehmern wichtige Vorträge und Informationen bezüglich des Themas \enquote{Digital Learning} näher gebracht werden. Das Thema ist vor allem in der heutigen digitalen Zeit ein sehr wichtiges Gebiet. Die Zielgruppen des Events sind Personen, die im Lehrbereich tätig sind, Studenten und Schüler, die vom digitalen Lernen betroffen sind oder auch Personen, die sich allgemein für das Thema interessieren. Auch das Thema \enquote{Remote Working} kann verknüpft werden, weswegen auch Führungskräfte zu der Zielgruppe zählen.
 
Das Konferenz-Konzept wird im Rahmen der Vorlesung \enquote{Integrationsseminar} von sieben Studenten umgesetzt. Im Folgenden werden die Rahmenbedingungen der Konferenz definiert. Diese findet von Donnerstag, den 23. Juni bis Sonntag, den 26. Juni statt wobei am ersten und letzten Tag keine Vorträge, Workshops und Debatten geplant sind. Wie bereits erwähnt fokussiert sich die Konferenz auf das Thema \enquote{Digital learning}. Hierzu werden mögliche Unterthemen und Sprecher überlegt, die an der Konferenz Vorträge halten könnten. Die Konferenz soll in den Räumen der D- und E-Gebäude der DHBW stattfinden. Es werden ungefähr 150 Teilnehmer erwartet, plus circa 20 Tagesgäste, die mit Essen und Trinken versorgt werden sollen. Die Kosten sollen sich allerdings im Rahmen halten, damit die Ticketkosten nicht 200 EUR pro Person übersteigen. Außerdem wird eine Online-Teilnahme angeboten, für die die Ticketkosten natürlich geringer werden. Hier werden weitere 100 Teilnehmer erwartet, wodurch mit einer Gesamtteilnahme von ungefähr 250 Personen gerechnet wird. Da die Konferenz-Teilnehmer aus aller Welt kommen, werden alle Dokumente, die die Teilnehmer über die Konferenz informieren sowie wichtige Hinweise liefern auf Englisch erstellt. Zu diesen Dokumenten gehört unter anderem ein FAQ, mit dem Fragen beantwortet werden, die bei den Teilnehmern aufkommen könnten, ein Anfahrtsplan zur DHBW vom Frankfurter und Stuttgarter Flughafen und welche Zusatzevents es in Mannheim und Heidelberg neben der Konferenz gibt. Außerdem wird ein Flyer mit den wichtigsten Informationen sowie eine Umfrage erstellt, die die Teilnehmer nach der Konferenz beantworten sollen. Da im Moment noch nicht abzusehen ist, wie die Corona-Lage im Konferenz-Jahr 2022 aussehen wird, ist es außerdem wichtig, ein Hygienekonzept für die Konferenz zu erstellen, damit die Veranstaltung regelkonform ablaufen kann.

\section{Themen und Sprecher}
Wie bereits erwähnt liegt der Fokus der Konferenz auf dem Thema \enquote{Digital Learning}. Im Anhang \vref{app:konferenzthemen} ist eine vollständige Liste möglicher Unterthemen zu finden, über die auf der Konferenz ein Vortrag gehalten werden könnte. Zu den Themenvorschlägen gehören unter anderem \enquote{Gamification}, \enquote{Digital Schools} und \enquote{Digital Learning @ DHBW}.

Des Weiteren wurden verschiedene Sprecher recherchiert, die möglicherweise einen Vortrag zum Thema \enquote{Digital Learning} halten könnten. Dazu gehört unter anderem der Chief Learning Officer der SAP, Max Wessel, sowie Jill Grinager vom Hasso-Plattner-Institut. Eine vollständige Liste der möglichen Sprecher kann in Tabelle\vref{tab:speaker} gefunden werden. Des Weiteren sollen noch Speaker gesucht werden, die einen Online-Vortrag halten und während der Konferenz nicht an der DHBW Vorort sind. 

\begin{table}[]
	\centering
	\resizebox{\textwidth}{!}{%
		\begin{tabular}{|l|l|l|}
			\hline
			\textbf{Name}          & \textbf{Unternehmen/Institution} & \textbf{Kontakt}              \\ \hline
			Max Wessel             & SAP                              & max.wessel@sap.com            \\ \hline
			Prof. Dr. Nagler       & DHBW                             & georg.nagler@dhbw-mannheim.de \\ \hline
			Jill Grinager          & HPI                              & jill.grinager@hpi.de          \\ \hline
			Johanna Schulz         & HPI                              & johanna.schulz@hpi.de         \\ \hline
			Maren Metz             & HFH                              & maren.metz@hamburger-fh.de    \\ \hline
			Dr. Michael Wache      &                                  & michael-wache.de              \\ \hline
			Prof. Dr. Andrea Honal & DHBW                             & andrea.honal@dhbw-mannheim.de \\ \hline
			Doro Moritz            & GEW                              & doro.moritz@gew-bw.de         \\ \hline
			Maire Landsberg        & BVMW                             & marie.landsberg@bvmw.de       \\ \hline
		\end{tabular}%
	}
	\caption{Vorgeschlagene Speaker für die MOBTS-Konferenz}
	\label{tab:speaker}
\end{table}

\section{Ablauf der Konferenz}
Der komplette Ablauf der Konferenz kann in Tabelle \vref{tab:konferenzablauf} angeschaut werden. 

\begin{table}[h]
	\centering
		\begin{tabular}{|l|l|l|}
			\hline
			\textbf{Datum}              & \textbf{Zeit} & \textbf{Event}                                                                                        \\ \hline
			23.06.2022                  & 16:30 - 20:30 & \begin{tabular}[c]{@{}l@{}}Begrüßung und Kennenlernen, \\ anschließend gemeinsames Essen\end{tabular} \\ \hline
			\multirow{7}{*}{24.06.2022} & 7:30          & Registration am Helpdesk                                                                              \\ \cline{2-3} 
			& 7:30 - 8:30   & Kaffee, Snacks to go                                                                                  \\ \cline{2-3} 
			& 8:30 - 9:30   & Eröffnungsrede                                                                                        \\ \cline{2-3} 
			& 9:45 - 13:00  & Sessions, workshops etc.                                                                              \\ \cline{2-3} 
			& 13:00 - 13:45 & Mittagspause                                                                                          \\ \cline{2-3} 
			& 13:45 - 17:15 & Sessions, workshops etc.                                                                              \\ \cline{2-3} 
			& 18:00         & Stadtführung                                                                                          \\ \hline
			\multirow{6}{*}{25.06.2022} & 8:30 - 9:00   & Eröffnungsrede                                                                                        \\ \cline{2-3} 
			& 9:00 - 13:00  & Sessions, workshops etc.                                                                              \\ \cline{2-3} 
			& 13:00 - 13:45 & Mittagspause                                                                                          \\ \cline{2-3} 
			& 13:45 - 16:00 & Sessions, workshops etc.                                                                              \\ \cline{2-3} 
			& 16:15 - 17:15 & Abschluss                                                                                             \\ \cline{2-3} 
			& 17:45         & Gemeinsames Essen                                                                                     \\ \hline
		\end{tabular}%
	\caption{Ablauf der Konferenz}
	\label{tab:konferenzablauf}
\end{table}
Die Teilnehmer kommen voraussichtlich am Donnerstag, den 23. Juni in Mannheim an. In der DHBW ist in dem Raum neben dem AudiMax dann bereits ein Tisch aufgebaut, bei dem sich die Teilnehmer registrieren können und eine Willkommens-Tüte mit verschiedenen Goodies und ihr Namensschild abholen können. Um 16:30 Uhr kann sich am Wasserturm oder an einer anderen Location zu einer kleinen Begrüßung und einem vorzeitigen Kennenlernen getroffen werden, wobei die Teilnahme hierfür selbstverständlich freiwillig ist. Hier sollte jedoch ein Ort gewählt werden bei dem genug Platz für alle Teilnehmer wäre.

Freitags haben ab 7:30 Uhr die Teilnehmer, die sich donnerstags noch nicht registriert hatten, noch einmal die Möglichkeiten dies nachzuholen. Gleichzeitig gibt es Kaffee und ein paar Snacks, bei denen die Teilnehmer noch einmal die Möglichkeit haben sich zu unterhalten oder zu frühstücken, bevor dann von 8:30 Uhr – 9:30 Uhr die Eröffnung der Konferenz folgt. Diese übernimmt Prof. Dr. Nagler im Raum Audi Max, da hier 200 Plätze zur Verfügung stehen und somit alle Teilnehmer einen Platz finden. Daraufhin folgt eine 15 Minuten Pause, bevor es dann mit den Vorträgen, Workshops und Debatten weitergeht. Diese dauern jeweils ungefähr eineinhalb Stunden. Mittags, von 12 Uhr bis 13 Uhr ist eine Mittagspause eingeplant. Hierfür werden im Raum über dem Audi Max verschiedene Finger Foods aufgebaut. Nach der Pause finden bis 17 Uhr weitere Sessions statt. Eine genauere Planung der verschiedenen Sessions ist hierbei erst möglich, wenn die Anzahl der Sprecher sowie die verschiedenen Themen feststehen. Erst dann ist ebenso eine endgültige Raumeinteilung möglich. In Tabelle %\vref{tab:sessionablauf}
ist allerdings dargestellt, wie ein möglicher Ablauf der Konferenz mit den Unterthemen, die sich im Rahmen des Konzeptes überlegt wurden, aussehen kann. Für abends (von 18 Uhr – 20:30 Uhr) ist ein freiwilliges Event für die Teilnehmer geplant. Dies könnte zum Beispiel eine Stadtführung sein, damit die Teilnehmer auch Mannheim kennen lernen können.
 
Samstags startet der Konferenz-Tag ab 8:30 Uhr. Hier folgt bis 9 Uhr eine kleine Begrüßung zum zweiten Tag und anschließend starten wieder die verschiedenen Sessions mit den Workshops, Präsentationen und Debatten mit Unterbrechung von 15 Minuten Pausen. 

Um einen Teil der Konferenz online ablaufen zu lassen, müssen in jedem Raum Kameras aufgestellt werden, die den Vortrag über eine Videokonferenz-Plattform wie zum Beispiel Zoom übertragen. Hier macht es Sinn, jemanden bei einem Laptop zu platzieren, der den Vortrag online verfolgt und die Fragen der digitalen Teilnehmer an den Vortragenden weiterleiten kann. 
Falls es zur Zeit der Konferenz immer noch Corona-Einschränkungen gibt, macht es möglicherweise Sinn, die verschiedenen Sessions so zu planen, dass die 15 Minuten Pausen zwischen den verschiedenen Sessions auf den selben Zeitpunkt fallen, um die Begegnungen geringer zu halten. Dadurch entsteht allerdings der Nachteil, dass eine Session erst endet, wenn eine andere bereits begonnen hat, wodurch die Teilnehmer in ihrer Wahl nicht mehr so flexibel sind.

\begin{table}[]
	\centering
		\begin{tabular}{|l|*{3}{>{\centering\arraybackslash}p{6cm}|}}
			\hline
			\textbf{Zeit} & \textbf{Raum A}                                    & \textbf{Raum B}                                                                                                   \\ \hline
			9:45          & \multirow{5}{*}{Digital Schools}                   & \multirow{5}{*}{Remote working}                                                                                   \\ \cline{1-1}
			10:00         &                                                    &                                                                                                                   \\ \cline{1-1}
			10:15         &                                                    &                                                                                                                   \\ \cline{1-1}
			10:30         &                                                    &                                                                                                                   \\ \cline{1-1}
			10:45         &                                                    &                                                                                                                   \\ \hline
			11:00         & Kaffeepause                                        & Kaffeepause                                                                                                       \\ \hline
			11:15         & \multirow{4}{*}{Digital Learning @ DHBW}           & \multirow{4}{*}{Duale Ausbildung International}                                                                   \\ \cline{1-1}
			11:30         &                                                    &                                                                                                                   \\ \cline{1-1}
			11:45         &                                                    &                                                                                                                   \\ \cline{1-1}
			12:00         &                                                    &                                                                                                                   \\ \hline
			12:15         & \multicolumn{2}{c|}{\multirow{3}{*}{Mittagspause}}                                                                                                                     \\ \cline{1-1}
			12:30         & \multicolumn{2}{c|}{}                                                                                                                                                  \\ \cline{1-1}
			12:45         & \multicolumn{2}{c|}{}                                                                                                                                                  \\ \hline
			13:00         & \multirow{4}{*}{Gamification}                      & \multirow{4}{*}{Motivation im digitalen Zeitalter}                                                                \\ \cline{1-1}
			13:15         &                                                    &                                                                                                                   \\ \cline{1-1}
			13:30         &                                                    &                                                                                                                   \\ \cline{1-1}
			13:45         &                                                    &                                                                                                                   \\ \hline
			14:00         & Kaffeepause                                        & Kaffeepause                                                                                                       \\ \hline
			14:15         & \multirow{5}{*}{\begin{tabular}[c]{@{}c@{}}Moocs / Online Corporate \\ Learning\end{tabular}} & \multirow{5}{*}{\begin{tabular}[c]{@{}c@{}}Social / Personal Development im \\ Online Zeitalter\end{tabular}}     \\ \cline{1-1}
			14:30         &                                                    &                                                                                                                   \\ \cline{1-1}
			14:45         &                                                    &                                                                                                                   \\ \cline{1-1}
			15:00         &                                                    &                                                                                                                   \\ \cline{1-1}
			15:15         &                                                    &                                                                                                                   \\ \hline
			15:30         & Kaffeepause                                        & Kaffeepause                                                                                                       \\ \hline
			15:45         & \multirow{6}{*}{Pädagogik für Dozenten}            & \multirow{6}{*}{\begin{tabular}[c]{@{}c@{}}Digitalisierung in der Hochschule \\ in Bezug auf Corona\end{tabular}} \\ \cline{1-1}
			16:00         &                                                    &                                                                                                                   \\ \cline{1-1}
			16:15         &                                                    &                                                                                                                   \\ \cline{1-1}
			16:30         &                                                    &                                                                                                                   \\ \cline{1-1}
			16:45         &                                                    &                                                                                                                   \\ \cline{1-1}
			17:00         &                                                    &                                                                                                                   \\ \hline
		\end{tabular}%
	\caption{Ein beispielhafter Ablauf eines Konferenztages.}
	\label{tab:sessionablauf}
\end{table}

\section{Hygienekonzept}
Da die Corona-Lage im Jahre 2022 noch nicht vorher gesagt werden kann ist es wichtig, das Hygienekonzept immer an die aktuellen Vorgaben anzupassen. Dabei sollte darauf geachtet werden wie viele Personen an der Konferenz teilnehmen dürfen, um eventuell auf eine mehrheitlich digitale Konferenz umzusteigen. Zum Zeitpunkt der Konferenzplanung besteht die Vorgabe für Messen bei sieben Quadratmeter pro Besucher. Des Weiteren sollten die allgemeinen Regeln eingehalten werden. So sollten die Besucher einen Mund-Nasen-Schutz tragen, wenn diese nicht an ihrem Platz sitzen oder gerade am Essen oder Trinken sind. Die Stühle in den Konferenzsälen sollten mit einem Abstand von mindestens 1,5 Meter Abstand zueinander aufgestellt werden. In Räumen, in denen es feste Sitze gibt, wie im Audi Max, sollten die Plätze die nicht belegt werden dürfen abgesperrt werden, um einen ausreichenden Abstand zu gewährleisten. Während den 15 Minuten Pausen zwischen zwei Sessions sollte der Raum durchgelüftet und gereinigt werden. Außerdem macht es Sinn, bei einer Ansammlung von einer großen Anzahl von Person auf einem engen Raum auch während der Session Fenster aufzumachen, falls es die Wetterlage erlaubt. Des Weiteren sollten ausreichend Desinfektionsmittelspender aufgestellt werden, damit die Teilnehmer sich die Hände desinfizieren können. Ebenso sollte ein separater Aus- und Eingang zu den Konferenzgebäuden zur Verfügung gestellt werden. So kann der eigentliche Eingang in Gebäude D weiterhin als Eingang verwendet werden, während die Tür beim Audi Max als Ausgang fungieren kann. Daher sollten auch verschiedene Schilder erstellt werden, die auf den Eingang und Ausgang hinweisen. Ebenfalls sollte es Schilder geben, die auf die verschiedenen Hygieneregeln hinweisen, wie die Mundschutzpflicht, die Husten- und Niesetikette, den ausreichenden Abstand, Händeschütteln vermieden werden sollte, und sich regelmäßig die Hände desinfiziert werden sollen. Eine weitere Vorsichtsmaßnahme ist es, ein Einbahnstraßensystem zu definieren und die Wege so abzukleben, damit die Teilnehmer den benötigten Abstand einhalten können. Auch sollten so viel Aktivitäten wie möglich draußen geplant werden, weswegen es Sinn macht, dass die Teilnehmer in den 15 Minuten Pausen möglicherweise nach draußen vor das Gebäude gehen, da hier auch mehr Platz zur Verfügung steht. 

\section{Catering}
Für die Verpflegung der Teilnehmer über die zwei Konferenz-Tage soll ein Catering bereitgestellt werden. Dieses wird im Raum über dem Audi Max aufgebaut werden, wodurch es nah an den Session-Räumen ist und somit leicht und schnell zu erreichen ist. Für das Catering wurde ein Angebot von Peters Party Service eingeholt. In dem Angebot bietet dieser Mittagessen von 11:30 - 14 Uhr an. Das Geschirr und Besteck sowie Servietten werden ebenfalls von ihm mitgebracht. An Tag eins hätte dieser als warme Option eine Kartoffelsuppe mit Gemüsestreifen und als kalte Optionen verschiedenes Fingerfood. Darunter Mini-Bifteki, Mini-Frikadellen, Falaffelbällchen und Käse-Obstspieße. Falls gewünscht würde er außerdem um 14:30 Uhr ein Blechkuchen-Mix anbieten. Am zweiten Tag wird als warme Speise eine ungarische Gulaschsuppe angeboten sowie als kalte Option belegte Partyrötchen, Mini-Frühlingsrollen, Blätterteighappen, Capresespieße, Frischkäse-Pasteten sowie gegrilltes, eingelegtes Gemüse. Als Dessert kann außerdem Panna Cotta oder Mousse au chocolat bestellt werden. Zusätzlich kann dieser verschiedene Getränke bereitstellen. Die Kosten des Caterings werden verrechnet und sind später Teil der Ticketkosten. Für die Essens-Kosten außerhalb der Konferenz, wenn zum Beispiel in einem der Restaurants gegessen wird, müssen die Teilnehmer selbst aufkommen.
Außerdem gibt es neben der DHBW einige Supermärkte bei denen die Teilnehmer sich etwas zu Essen kaufen können sowie eine Kaffeerösterei, \enquote{AGATA Rösterei \& Cafe}. 

\section{Kostenkalkulation}
Die Ticketkosten berechnen sich aus den Konferenzkosten. Aktuell ist es noch nicht möglich eine genaue Kostenkalkulation durchzuführen, da noch kein festes Catering und kein fester Plan für die Konferenz besteht. So müsste für das gemeinsame Essen Gehen möglicherweise ein Restaurant wie zum Beispiel Lindbergh gebucht werden, um alle Teilnehmer unter zu bekommen. Diese Kosten sollten hierbei ebenfalls in die Ticketkosten mit einberechnet werden. Andere Kosten wie die Stadtführung und das Essen in Restaurants dagegen wird von den Teilnehmern selbst übernommen. 

Falls sich beim Catering für das Angebot von Peters Party Service entschieden wird, belaufen sich die Kosten für 200 Teilnehmer\footnote{Da im Angebot von 200 Teilnehmern ausgegangen wird, wird in der Berechnung ebenfalls von 200 Teilnehmern ausgegangen. Die Kosten können demnach entsprechend variieren.} auf 9.734,20 EUR, was pro Person einem Preis von 48,67 EUR entspricht. Bei der Berechnung wurde damit gerechnet, dass 800 0,7 Liter Flaschen Mineralwasser getrunken werden, da Mineralwasser der Mittelwert der Getränke ist und noch nicht vorhergesehen werden kann, was und wie viel getrunken wird. Daher wird auch von einem Durchschnittswert von 4 Flaschen Wasser pro Tag pro Teilnehmer ausgegangen. Außerdem fallen für Zoom Webinar, falls die Konferenz darüber ausgestrahlt wird, weitere Kosten an. Zoom Webinar kostet 37 EUR pro Monat für max. 100 Teilnehmer, was planmäßig ausreichen sollte. Das bedeutet, dass im Moment Kosten von insgesamt 9.771,20 EUR anfallen. Pro Person sind das aktuell wiederum 48,86 EUR. 

% Please add the following required packages to your document preamble:
% \usepackage{graphicx}
\begin{table}[]
	\centering
	\resizebox{\textwidth}{!}{%
		\begin{tabular}{|l|l|r|r|}
			\hline
			\multicolumn{2}{|l|}{\textbf{Kosten}}                & \textbf{für 200 Teilnehmer} & \textbf{für 100 Teilnehmer} \\ \hline
			Anlieferung       &                                  & 90 EUR                        & 90 EUR                       \\ \hline
			Mittagessen Tag 1 & à 11,90 EUR                        & 2.380 EUR                     & 1.190 EUR                     \\ \hline
			Mittagessen Tag 2 & à 14,90 EUR                        & 2.980 EUR                     & 1.490 EUR                     \\ \hline
			Getränke          & 800 Mineralwasser 0,7 l à 2,10 EUR & 1.680 EUR                     & 840 EUR                       \\ \hline
			Gläser            &                                  & 250 EUR                       & 250 EUR                       \\ \hline
			Blechkuchen-Mix   & 400 Stück à 2 EUR                  & 800 EUR                       & 400 EUR                       \\ \hline
			Summe             &                                  & 8.180 EUR                     & 4.260 EUR                     \\ \hline
			Mehrwertsteuer    & 19 \%                            & 1.554,2 EUR                   & 809,4 EUR                     \\ \hline
			Summe             &                                  & 9.734,2 EUR                   & 5.069,4 EUR                   \\ \hline
			Pro Person        &                                  & 48,67 EUR                     & 50,69 EUR                     \\ \hline
		\end{tabular}%
	}
	\caption{Kostenkalkulation für das Catering.}
	\label{tab:cateringkosten}
\end{table}

\section{Umfrage}
Um nach der Konferenz ein Bild darüber zu bekommen, wie erfolgreich diese war, wurde eine Umfrage erstellt. Diese können die Teilnehmer nach der Konferenz beantworten. Die Umfrage kann im Anhang ... gesehen werden. Dazu gehören persönliche Fragen, um einen Überblick zu haben, von welchem Kontinent der befragte kommt, welches Geschlecht dieser hat und welche Position. Anschließend sind verschiedene Fragen zur Konferenz aufgelistet. Hier werden die Teilnehmer zuerst über die DHBW und ihre Ausstattung befragt. Unter anderem wie sie die technische Ausstattung und die Gastfreundschaft der DHBW fanden. Weiterhin werden die Teilnehmer über die angebotenen Sessions befragt, darunter wie sie diese und die Speaker fanden sowie wie ihnen die Agenda der Sessions gefallen hat. Ebenso werden die Teilnehmer befragt wie sie die Aktivitäten außerhalb der Konferenz fanden und ob diese noch einmal an einer Konferenz teilnehmen würden, die von der DHBW veranstaltet wird. Zum Schluss kann noch ein Kommentar abgegeben werden, falls der Befragte denkt es wäre irgendwas erwähnenswert.  

\section{Administrative Aufgaben}
Um das Konferenz-Konzept umsetzen zu können, gibt es einige administrative Aufgaben die umgesetzt werden müssen. Unter anderem müssen die verschiedenen Goodies (zum Beispiel Kugelschreiber, Blöcke, Mundschutz etc.) für die Welcome-Tüte zusammengestellt sowie die Tüten organisiert werden. Außerdem müssen Namensschilder für alle Teilnehmer erstellt werden, die diese am ersten Tag beim Helpdesk abholen können. Der Helpdesk muss dafür vorbereitet werden. Außerdem müssen mindestens zwei Personen die ganze Zeit am Helpdesk vertreten sein, um bei Fragen den Teilnehmern zu helfen. Außerdem müssen verschiedene Schilder erstellt werden. Dazu gehören Richtungsschilder zum Helpdesk, der Kantine, den Toiletten und Raumschilder. Für Corona müssen außerdem Ausgangs- und Eingangsschilder erstellen, sowie Wege ab geklebt werden. Am Eingang macht es außerdem Sinn ein Schild mit einer Raumübersicht sowie Session-Übersicht zu geben, wann welcher Vortrag in welchem Raum stattfindet. Dazu könnte eine kleinerer Übersicht am Helpdesk für die Teilnehmer ausgegeben werden. Ebenfalls muss der Raum über dem Audi Max, der als Kantine dienen soll, vorbereitet werden, indem die Tische und Stühle aufgestellt werden. Ebenso müssen im Falle von Corona genug Desinfektionsspender zur Verfügung gestellt werden sowie Personen engagiert werden, die in den 15 Minuten Pausen zwischen den Sessions den Raum lüften und reinigen. Auch Personen, die sich um mögliche IT-Probleme während der Konferenz kümmern müssen vorhanden sein. 

\section{Weitere Informationen}
Damit die Teilnehmer Mannheim besser kennen lernen können, werden ihnen bereits vorab verschiedene Informationen zur Verfügung gestellt. Darunter welche Sehenswürdigkeiten es in Mannheim und Heidelberg gibt. Hier gibt es Links zu Stadtführungen mit und ohne Führer, sowie Informationen zu den verschiedenen Sehenswürdigkeiten, wie die Öffnungszeiten, Ticketpreise und die Adresse. Außerdem wurde ein Dokument zusammengestellt, in welchem die verschiedenen Erfindungen aufgelistet werden, die aus Mannheim kommen. Dazu gehören unter anderem das erste Automobil und das Spaghettieis. Des Weiteren gibt es ein FAQ Dokument, in dem mögliche Fragen der Teilnehmer beantwortet werden. Ebenfalls wurde ein Flyer erstellt, in dem die wichtigsten Informationen zusammengetragen sind, damit die Teilnehmer alle Informationen auf einem Blick haben. (Bild Flyer)
