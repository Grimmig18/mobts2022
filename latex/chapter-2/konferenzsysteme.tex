\chapter{Konferenzsysteme}
Zur Verrichtung von Onlinekonferenzen können unterschiedliche Konferenzsysteme verwendete werden.
Das folgende Kapitel gibt zuerst einen Überblick über die Grundlagen von Onlinekonferenzsystemen.
Anschließend werden die Konferenzsysteme \enquote{Zoom}, \enquote{Blackboard Collaborate} und \enquote{BigBlueButton} detailliert vorgestellt.
Neben der detaillierten Vorstellung von Zoom, Blackboard Collaborate und  BigBlueButton werden Konferenzsysteme andere Hersteller kurz vorgestellt, um einen Überblick über verfügbare Konferenzlösungen zu geben.
Die detailliert vorgestellten Onlinekonferenzlösungen werden zudem auf Basis verschiedenere Merkmale wie Kosten, Skalierbarkeit oder Leistung vergleichend gegenübergestellt.
Der Vergleich kann verwendet werden, um ein geeignetes Konferenzsystem für die gegebenen Anforderungen zu finden.

\section{Grundlagen zu Konferenzsystemen}
Konferenzsysteme lassen sich in die Kategorien \textit{Videokonferenzsysteme} und \textit{Webkonferenzsysteme} einteilen.
\autocite[Vgl.][]{M_Straub.o.J.}
Auch wenn die Begriffe synonym verwendet werden, steht bei Videokonferenzen der Austausch per Video im Vordergrund.
Bei Webkonferenzen dagegen steht das Teilen von Anwendungen und das gemeinsame Zusammenarbeiten im Vordergrund.
\autocite[Vgl.][]{M_Straub.o.J.}
\\
Um an Onlinekonferenzen teilnehmen zu können, müssen einige Grundvoraussetzungen erfüllt werden.
Als Erstes wir ein Gerät benötigt, welches die Teilnahme ermöglicht.
Das kann ein Computer oder auch ein mobiles Endgerät sein.
Wenn an der Konferenz via Video teilgenommen werden soll, wird zusätzlich eine Kamera benötigt.
Mobile Endgeräte haben diese meist direkt integriert.
Für eine Audioteilnahme werden ein Mikrofon und Lautsprecher benötigt.
Statt der einzelnen Audiokomponenten kann auch ein Headset verwendet werden.
\autocite[Vgl.][]{M_Straub.o.J.}
\\
Neben den Hardwarekomponenten wird eine Internetverbindung benötigt.
Für Videokonferenzen sollte eine Bandbreite von mindestens 200 Kbits/s für den Up- und Download zur Verfügung stehen.
Des Weiteren wird ein geeignetes Onlinekonferenzsystem benötigt.
Dazu gibt es eine große Auswahl an Onlinekonferenzwerkzeugen.
\autocite[Vgl.][]{M_Straub.o.J.}
\\
Mit Videokonferenzen kann eine effiziente Zusammenarbeit mit Personen auf der ganzen Welt erreicht werden.
Digitale Treffen bieten viele Möglichkeiten, die ein persönliches Treffen in einigen Bereichen ersetzen können.
\autocite[Vgl.][]{M_Mierke.2020}
\\
Onlinekonferenzen helfen zudem bei der Einsparung von Reisezeit sowie der Einsparung von gefahrenen oder geflogenen Kilometern.
Auch können Personen an Konferenzen teilnehmen, die gleichzeitig Kinder betreuen müssen oder denen die Einreise verwehrt worden ist.
\autocite[Vgl.][]{M_Sladek.2020}
\\
In der Wirtschaft spielen Online-Seminare und Konferenzen eine große Rolle.
Mitarbeiter und Führungskräfte können durch den Einsatz von digitalen Konferenzen und Schulungen weitergebildet werden.
Onlineschulungen helfen dabei, dass das Personal keine zusätzliche Arbeitszeit benötigt und keine Reisekosten hat, um an einer Schulung teilnehmen zu können.
\autocite[Vgl.][]{M_GrunderkucheRedaktion.2021}
\\
Im IT-Bereich nutzen Hersteller eigene Onlineschulungen oft auch dazu, um Kunden anzuwerben.
\autocite[Vgl.][]{M_GrunderkucheRedaktion.2021}
Es gibt für das Abhalten von Onlinekonferenzen verschieden Möglichkeiten:\\
Konferenzen können dabei als Videokonferenz, als Liveevent oder als vorher aufgenommene Session abgehalten werden.
Je nach gewählter Präsentationsmethode können Zuschauer unterschiedlich mit der Konferenz interagieren.
\autocite[Vgl.][]{M_Sladek.2020}
\\
Sollen Diskussionen oder Workshops durchgeführt werden, können nur Videokonferenzen abgehalten werden.
Teilnehmer sollten für eine erfolgreiche Teilnahme an Workshops eine gute Audioqualität haben.
Zudem ist es sinnvoll, dass Onlinediskussionen moderiert werden.
Für eine bessere Zuordnung der Sprecher kann es hilfreich sein, dass Teilnehmer ein Profilbild von sich bereitstellen oder per Video an der Diskussion teilnehmen.
Diese Art von Onlinekonferenzen eignet sich vor allem für geringe Teilnehmerzahlen.
\autocite[Vgl.][]{M_Sladek.2020}
\\
Eine andere Möglichkeit für Onlinekonferenzen ist das Livestreamen von Events.
Dabei wird ein Event mit einer geringen Verzögerung an eine große Anzahl an Zuschauer übertragen.
\autocite[Vgl.][]{M_Sladek.2020}
\\
Ein Nachteil von Livestreaming ist die Voraussetzung einer stabilen und schnellen Internetverbindung.
\autocite[Vgl.][]{M_Maciej.2016}
Wenn eine stabile Internetverbindung nicht gewährleistet werden kann, bietet es sich an, vorher Inhalte aufzunehmen und als Video bereitzustellen.
Zudem können Teilnehmer so selber entscheiden, zu welchem Zeitpunkt sie welche Inhalte konsumieren.
Es kann zudem dafür gesorgt werden, dass Inhalte barrierefrei bereitgestellt werden.
So können zum Beispiel vorher Untertitel erzeugt und für die Videos bereitgestellt werden.
\autocite[Vgl.][]{M_Sladek.2020}
\\
Zwischenmenschliche Kommunikation kann jedoch nicht komplett durch digitale Konferenzen ersetzt werden.
Das Kennenlernen anderer Personen und das Schließen von neuen Bekanntschaften ist im Rahmen von Onlinekonferenzen fast nicht möglich.
Es kann für einige Teilnehmer schwierig sein, andere Konferenzteilnehmer per Chat anzuschreiben und ein Gespräch aufzubauen.
\autocite[Vgl.][]{M_Sladek.2020}
\\
Anhand dieser Informationen wird deutlich, dass Onlinekonferenzen einige Vorteile bieten, aber auch Nachteile haben.
Sie eignen sich vor allem dazu, einem großen Publikum über zum Teil große Distanzen neue Inhalte zu vermitteln, dabei bleibt die zwischenmenschliche Kommunikation aber oft auf der Strecke.
Für die digitale Fernlehre bieten Onlinekonferenzen und Seminare jedoch ein großes Potential.
Entscheidend für die digitale Lehre ist der richtige Einsatz der gegebene Werkzeuge.

\section{Konferenzsysteme vorgestellt}
\label{sec:konferenzsysteme_vorgestellt}
\subsection{Zoom}
Zoom ist eine von der \textit{Zoom Video Communications Inc.} bereitgestellte Lösung für Onlinekonferenzen.
\autocite[Vgl.][]{M_Zoom.o.J.b}
In der kostenlosen Variante von Zoom können bis zu 100 Teilnehmer an einer Videokonferenz teilnehmen.
Sobald sich mehr als drei Personen in einer Sitzung befinden, wird die maximale Sitzungsdauer auf 40 Minuten begrenzt.
Um eine längere Sitzungsdauer zu ermöglichen, ist es notwendig, die kostenpflichtige Version von Zoom zu verwenden.
Damit kann die Sitzungsdauer auf bis zu 24 Stunden verlängert werden. Die Kosten dafür betragen 13,99 Euro pro Monat pro Moderator.
Ein Moderator stellt bei Zoom einen Gastgeber für eine Onlinekonferenz dar.
Um einen Schutz vor unberechtigtem Betreten von Meetings zu ermöglichen, können Zoommeeting mit einem Passwortschutz versehene werden.
So können nur Teilnehmer mit dem entsprechenden Passwort an einer Onlinekonferenz teilnehmen.
\autocite[Vgl.][]{M_Mierke.2020}
\\
Neben den genannten Funktionen können Teilnehmer von Zoommeetings bis zu 49 Videos pro Bildschirm sehen.
Das Teilnehmerlimit für eine Sitzung liegt bei 1000 Teilnehmern.
Um Inhalte wie Präsentationen zu teilen, bietet Zoom Teilnehmern die Möglichkeit, ihren Bildschirm freizugeben.
Die Funktion kann dabei von mehreren Teilnehmern gleichzeitig genutzt werden, sodass in einem Meeting mehrerer Bildschirme zur selben Zeit geteilt werden können.
Um ein Meeting auch für Teilnehmer bereitzustellen, die Internetprobleme haben oder die verhindert sind, können Meetings aufgezeichnet werden.
\autocite[Vgl.][]{M_Zoom.o.J.b}
\\
Teilnehmer können mit verschiedenen Endgeräten an Zoommeetings teilnehmen.
Dazu kann eine entsprechende Anwendung auf einem Computer oder mobilen Endgerät installiert werden.
Zoom bietet Desktop-Clients für die Betriebssyteme Windows, MacOS und Linux an.
Für mobile Endgeräte wie Tablets oder Smartphones bietet Zoom Applikationen für die Betriebssysteme Android und iOS an.
Neben der Nutzung des Zoom-Clients auf einem Endgerät können Teilnehmer auch über ihren Webbrowser an Meetings teilnehmen.
\autocite[Vgl.][]{M_Zoom.o.J.}

\subsection{Blackboard Collaborate}
Blackboard Collaborate ist eine von \textit{Blackboard Inc.} bereitgestellte Onlinekonferenzlösung, die vor allem auf die Onlinelehre abzielt.
\autocite[Vgl.][]{M_Blackboard.o.J.}
Blackboard Collaborate gibt an, besonders für die Bildung gemacht zu sein und auf die Bedürfnisse von Lehrenden und Lernenden angepasst zu sein.
Dazu bietet Blackboard Collaborate die Möglichkeit, direkt aus dem Webbrowser heraus nutzbar zu sein.
Der Download eines besonderen Clients wird dabei nicht benötigt.
Wie bei Zoom gibt es auch bei Blackboard Collaborate verschiedene Lizenzen, die mehr oder weniger Funktionen zur Verfügung stellen.
Es wird zwischen der Enterprise und der Departmentlizenz unterschieden.
Die Departmentlizenz bietet die Möglichkeit, dass bis zu 500 Teilnehmer an einer Konferenz teilnehmen können.
Zusätzlich können bis zu 500GB an Dateien gespeichert werden. Die Sitzungszeit ist auf 1000000 Minuten begrenzt.
Die Enterpriselizenz bietet wie die Departmentlizenz maximal 500 Teilnehmern die Möglichkeit, an einer Onlinesitzung teilzunehmen.
Die Speicher- und Sitzungszeitbegrenzungen sind jedoch Variable.
Die Kosten für die Departmentlizenz betragen 9000 Dollar pro Jahr, für die Enterpriselizenz wird ein individueller Preis gezahlt, der von den gewünschten Funktionen abhängt.
\autocite[Vgl.][]{M_Blackboard.o.J.}
\\
Das Blackboard Collaborate kann auf verschiedene Weisen genutzt werden.
Es gibt die Möglichkeit, wie in einem Klassenraum den Teilnehmern eine Anwendung zu teilen und Dinge zu präsentieren.
Zudem können Teilnehmer in kleineren Gruppen zusammenarbeiten.
\autocite[Vgl.][]{M_Blackboard.o.J.}
\\
Auch besteht die Möglichkeit, Umfragen zu erstellen oder virtuell die Hand zu heben.
Auf diese Weise können Lehrende mit den Lernenden interagieren und Feedback erhalten.
Neben der Möglichkeit einer Sitzung mit einer Internetverbindung beizutreten, bietet Blackboard Collaborate die Möglichkeit, per Telefon an einer Sitzung teilzunehmen.
Dabei kann jedoch nur die Audiofunktion genutzt werden.
Falls die eigene Audioeingabe nicht funktionieren sollte, kann der integrierte Chat zur Kommunikation zwischen den Teilnehmern einer Sitzung genutzt werden.
Die Nutzung des digitalen Whiteboards kann Lehrenden dabei helfen, Dinge wie an einer Tafel zu digitalisieren.
\autocite[Vgl.][]{M_NorthernIllinoisUniversity.o.J.}

\subsection{BigBlueButton}
BigBlueButton ist wie Blackboard Collaborate ein Onlinekonferenzwerkzeug mit dem Fokus auf digitalem Lernen.
Genau wie Blackboard Collaborate ist BigBluebutton vor allem ein Webkonferenzsystem, welches in einem Webbrowser verwendet werden kann.
BigBlueButton ist zudem ein Open Source Projekt.
\autocite[Vgl.][]{M_BigBlueButton.o.J.b}
\\
Bei Open Source Projekten ist der Programmcode für alle Menschen einsehbar.
Das bietet den Vorteil, dass Menschen aus der ganzen Welt an dem Projekt mitentwickeln können.
Durch Open Source können Fehler schneller entdeckt und behoben werden.
Zudem können Programme an individuelle Bedürfnisse angepasst werden.
\autocite[Vgl.][]{M_RedHat.o.J.}
\\
BigBlueButton bietet wie Blackboard Collaborate verschiedene Nutzungsmöglichkeiten.
Lehrende können beispielsweise ein digitales Whiteboard nutzen, um Dinge zu visualisieren.
Die Zeichnungen können von den Sitzungsteilnehmern in Echtzeit verfolgt werden.
Um zwischenmenschliche Kommunikation zu ermöglichen, können alle Teilnehmer per Videofeed an der Sitzung teilnehmen.
Es gibt dabei kein Limit, wie viele Teilnehmer ihre Kamera nutzen können.
Das einzige Limit ist die Internetbandbreite der Teilnehmer.
\autocite[Vgl.][]{M_BigBlueButton.o.J.}
\\
BigBlueButton bietet neben dem Teilen von Inhalten und halten von Videokonferenzen die Möglichkeit zu chatten, Emojis zu verwenden, Umfragen zu starten sowie zusammen an Whiteboards zu arbeiten.
Zudem können sogenannte \enquote{Breakout Rooms} erstellt werden, um so in kleineren Gruppen zusammenarbeiten zu können.
\autocite[Vgl.][]{M_BigBlueButton.o.J.}
\\
Im Gegensatz zu Zoom ist BigBlueButton kein direkt nutzbarere Service.
Um BigBlueButton nutzen zu können, muss ein eigener Server verwendet werden, auf dem die Software installiert wird.
Auch ist es notwendig, die gewünschten Einstellungen selbst vorzunehmen, sodass auch hier Zeit und personelle Ressourcen benötigt werden.
Der Vorteil an einer selbst gehosteten Lösung ist jedoch die Kontrolle über die anfallenden Daten.
Zudem kann der Serverstandort selbst bestimmt werden.
So kann die Verarbeitung von personenbezogenen Daten in Deutschland oder der Europäischen Union gewährleistet werden.
\autocite[Vgl.][]{M_Klicksafe.o.J.}

\subsection{Weitere Onlinekonferenzsysteme}
Neben den vorgestellten Lösungen für Onlinekonferenzen existiert eine Vielzahl weiterer Anwendungen.
Diese Werkzeuge sind dabei auf die unterschiedlichen Anwendungsfälle und Bedürfnisse der Benutzer angepasst.
Im Folgenden wird eine kleine Übersicht über weitere Onlinekonferenzsysteme und ihre Einsatzmöglichkeit gegeben.
Die Werkzeuge werden dabei nicht so detailliert behandelt wie die in \autoref{sec:konferenzsysteme_vorgestellt} vorgestellten Lösungen.
\\
Einige Hersteller der im Folgenden genannten Werkzeuge bieten aufgrund von Corona eigentlich kostenpflichtige Funktionen zur kostenlosen Nutzung an.
\autocite[Vgl.][]{M_Straub.o.J.}

\subsubsection{Skype}
Um Skype nutzen zu können, wird ein Skype-Account benötigt.
Es können kostenlos Anrufe mit bis zu 50 Teilnehmern durchgeführt werden.
Ein Nachteil von Skype sind häufige Störungen in der Übertragung.
\autocite[Vgl.][]{M_Straub.o.J.}
\\
Neben einer privaten Lizenz könne Firmen \textit{Skype for Business} verwenden.
\autocite[Vgl.][]{M_Microsoft.o.J.}

\subsubsection{Google Meet}
\textit{Meet} ist eine von Google zur Verfügung gestellte Videokonferenzlösung.
Sie funktioniert über den Webbrowser und ist kostenpflichtig.
Dazu kann eine Unternehmenslizenz erworben werden.
Nur Teilnehmer mit einer Lizenz können Meetings organisieren, die Teilnahme an organisierten Meetings kann jedoch auch ohne eine eigene Lizenz erfolgen.
Meet ist in Googles \textit{G Suite} integriert, daher lassen sich Termine und Kontakte einfach importieren
\autocite[Vgl.][]{M_Straub.o.J.}

\subsubsection{Microsoft Teams}
Microsoft Teams beinhaltet neben der Videokonferenzfunktion einen Chat sowie die Möglichkeit, Dateien auszutauschen.
Zudem lassen sich Microsoft-Anwendungen wie \textit{Excel}, \textit{Word} und \textit{Power Point} direkt in Teams nutzen.
Dateien werden über \textit{Microsoft Sharepoint} allen Teilnehmern einer Gruppe bereitgestellt.
Die Planung von Meetings kann mit \textit{Outlook} und dem Werkzeug \enquote{Planner} erfolgen.
Teams ist für bis zu 300 Teilnehmer kostenlos, es können jedoch auch kostenpflichtige Unternehmenslizenzen erworben werden.
\autocite[Vgl.][]{M_Straub.o.J.}

\subsubsection{Bitrix 24}
Bitrix 24 eignet sich vor allem für kleine Firmen und bietet umfassende Möglichkeiten zur Projekt- und Aufgabenplanung.
Es gibt sechs unterschiedliche Lizenzen, von denen eine kostenlos ist.
In der kostenlosen Version können bis zu zwölf Teilnehmer an einem Meeting teilnehmen.
\autocite[Vgl.][]{M_Straub.o.J.}

\subsubsection{Mikogo}
Mikogo ist hauptsächlich für das Teilen von Bildschirmen geeignet.
Es bietet sowohl eine kostenlose als auch eine kostenpflichtige Version an.
Die kostenlose Version ermöglicht es mit bis zu 25 Teilnehmern zu kommunizieren.
Mikogo ermöglicht es zudem Dokumente zu teilen, Sitzungen aufzuzeichnen und ein virtuelles Whiteboard zur Verfügung zu stellen.
Es ist vollständig webbasiert, was den Download zusätzlicher Software überflüssig macht.
\autocite[Vgl.][]{M_Straub.o.J.}

\section{Vergleich der Konferenzsysteme}
Hier eine Tabelle rein und kurz was schreiben\\
TODO: ZU DEN EINZELNEN TOOLS AUF JEDEN FALL NOCH WAS SCHREIBEN