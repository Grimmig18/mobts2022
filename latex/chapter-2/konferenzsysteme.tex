\chapter{Konferenzsysteme}
Zur Verrichtung von Online Konferenzen können unterschiedliche Konferenzsysteme verwendete werden.
Im Folgenden werden Grundlagen zu online Konferenzsystemen erklärt.
Anschließend werden exemplarisch die Konferenzsysteme \enquote{Zoom}, \enquote{BigBlueButton} und \enquote{Blackboard Collaborate} vorgestellt.
Zudem werden die Systeme auf Basis verschiedenere Merkmale wie Kosten, Skalierbarkeit oder Leistung vergleichend gegenübergestellt.
Der Vergleich kann dabei helfen ein geeignetes Konferenzsystem, für die gegebenen Anforderungen zu finden.

\section{Grundlagen zu Konferenzsystemen}
Mit Videokonferenzen kann eine effiziente Zusammenarbeit, mit Personen auf der ganzen Welt, erreicht werden.
Digitale Treffen bieten viele Möglichkeiten, die ein persönliches Treffen in einigen Bereichen ersetzen können.
\autocite[Vgl.][]{M_Mierke.2020}
%(Vgl. \url{https://www.heise.de/tipps-tricks/Videokonferenz-Tools-im-Ueberblick-4688243.html}, Abgerufen am 26.02.2021).
\\
Onlinekonferenzen helfen zudem bei der Einsparung von Reisezeit, sowie der Einsparung von gefahrenen oder geflogenen Kilometern.
Auch können Personen an Konferenzen teilnehmen, die gleichzeitig Kinder betreuen müssen oder denen die Einreise verwehrt ist.
\autocite[Vgl.][]{M_Sladek.2020}
%(Vgl. \url{https://hochschulforumdigitalisierung.de/de/blog/online-tagungen-organisieren}, Abgerufen am 26.02.2021 ).
\\
Es gibt für das Abhalten von Onlinekonferenzen verschieden Möglichkeiten:\\
Konferenzen können dabei als Videokonferenz, als Liveevent oder als vorher aufgenommene Session abgehalten werden.
Je nach gewählter Präsentationsmethode können Zuschauer unterschiedlich mit der Konferenz interagieren.
\autocite[Vgl.][]{M_Sladek.2020}
%(Vgl. \url{https://hochschulforumdigitalisierung.de/de/blog/online-tagungen-organisieren}, Abgerufen am 26.02.2021 ).
\\
Sollen Diskussionen oder Workshops durchgeführt werden, können nur Videokonferenzen abgehalten.
Teilnehmer sollten für eine erfolgreiche Teilnahme an Workshops eine gute Audioqualität haben.
Zudem ist es sinnvoll, dass Onlinediskussionen moderiert werden.
Für eine bessere Zuordnung der Sprecher kann es hilfreich sein, dass Teilnehmer ein Profilbild von sich bereitstellen oder per Video an der Diskussion teilnehmen.
Diese Art von Onlinekonferenzen eignet sich vor allem für geringe Teilnehmerzahlen.
\autocite[Vgl.][]{M_Sladek.2020}
%(Vgl. \url{https://hochschulforumdigitalisierung.de/de/blog/online-tagungen-organisieren}, Abgerufen am 26.02.2021 ).
\\
Eine andere Möglichkeit für Onlinekonferenzen ist das Livestreamen von Events.
Dabei wird ein Event mit einer geringen Verzögerung an eine große Anzahl an Zuschauer übertragen.
\autocite[Vgl.][]{M_Sladek.2020}
%(Vgl. \url{https://hochschulforumdigitalisierung.de/de/blog/online-tagungen-organisieren}, Abgerufen am 26.02.2021 ).
\\
Ein Nachteil von Livestreaming ist die Voraussetzung einer stabilen und schnellen Internetverbindung.
\autocite[Vgl.][]{M_Maciej.2016}
%(Vgl. \url{https://www.giga.de/extra/livestream/specials/was-ist-streaming-erklaerung-fuer-dummys/}, Abgerufen am 26.02.2021)
Wenn eine stabile Internetverbindung nicht gewährleistet werden kann, bietet es sich an, vorher Inhalte aufzunehmen und als Video bereitzustellen.
Zudem können Teilnehmer so selber entscheiden, zu welchem Zeitpunkt sie welche Inhalte konsumieren.
Es kann zudem dafür gesorgt werden, dass Inhalte barrierefrei bereitgestellt werden.
Es können so zum Beispiel vorher Untertitel erzeugt werden und für die Videos bereitgestellt werden.
\autocite[Vgl.][]{M_Sladek.2020}
%(Vgl. \url{https://hochschulforumdigitalisierung.de/de/blog/online-tagungen-organisieren}, Abgerufen am 26.02.2021 ).
\\
Zwischenmenschliche Kommunikation kann jedoch nicht durch digitale Konferenzen ersetzt werden.
Das Kennenlernen anderer Personen und das Schließen von neuen Bekanntschaften ist im Rahmen von Onlinekonferenzen fast nicht möglich.
So kann es für einige Teilnehmer schwierig sein, andere Konferenzteilnehmer per Chat anzuschreiben und ein Gespräch aufzubauen.
\autocite[Vgl.][]{M_Sladek.2020}
%(Vgl. \url{https://hochschulforumdigitalisierung.de/de/blog/online-tagungen-organisieren}, Abgerufen am 26.02.2021 ).
\\
Daran wird deutlich, dass Onlinekonferenzen einige Vorteile bieten, aber auch Nachteile haben.
Sie eignen sich um einem großen Publikum über zum Teil große Distanzen neue Inhalte zu vermitteln, dabei bleibt die zwischenmenschliche Kommunikation aber oft auf der Strecke.

\section{Konferenzsysteme vorgestellt}
\label{sec:konferenzsysteme_vorgestellt}
\subsection{Zoom}
Zoom ist eine von der \textit{Zoom Video Communications Inc.} bereitgestellte Lösung für Onlinekonferenzen.
\autocite[Vgl.][]{M_Zoom.o.J.b}
%(Vgl. \url{https://zoom.us/de-de/meetings.html}, Abgerufen am 26.02.2021).
In der kostenlosen Variante von Zoom können bis zu 100 Teilnehmer an einer Videokonferenz teilnehmen.
Sobald sich mehr als drei Personen in einer Sitzung befinden, wird die maximale Sitzungsdauer auf 40 Minuten begrenzt.
Um eine längere Sitzungsdauer zu ermöglichen ist es notwendig die kostenpflichtige Version von Zoom zu verwenden.
Damit kann die Sitzungsdauer auf bis zu 24 Stunden verlängert werden. Die Kosten dafür betragen 13,99 Euro pro Monat pro Moderator.
Ein Moderator stellt bei Zoom einen Gastgeber für eine Onlinekonferenz dar.
Um einen Schutz vor unberechtigtem Betreten von Meetings zu ermöglichen, können Zoommeeting mit einem Passwortschutz versehene werden.
So können nur Teilnehmer mit dem entsprechenden Passwort an einer Onlinekonferenz teilnehmen.
\autocite[Vgl.][]{M_Mierke.2020}
%(Vgl. \url{https://www.heise.de/tipps-tricks/Videokonferenz-Tools-im-Ueberblick-4688243.html}, Abgerufen am 26.02.2021).
\\
Neben den Funktionen können Teilnehmer von Zoommeetings bis zu 49 Videos pro Bildschirm sehen.
Das Teilnehmerlimit liegt bei 1000 Teilnehmern. Um Inhalte wie Präsentationen zu teilen, bietet Zoom Teilnehmern die Möglichkeit ihren Bildschirm freizugeben.
Die Funktion kann dabei von mehreren Teilnehmern gelichzeitig genutzt werden, sodass in einem Meeting mehrerer Bildschirme zur selben Zeit geteilt werden können.
Um ein Meeting auch für Teilnehmer bereitzustellen, die Internetprobleme haben oder die verhindert sind, können Meetings aufgezeichnet werden.
\autocite[Vgl.][]{M_Zoom.o.J.b}
%(Vgl. \url{https://zoom.us/de-de/meetings.html}, Abgerufen am 26.02.2021).
\\
Teilnehmer können mit verschiedenen Endgeräten an Zoommeetings teilnehmen.
Es gibt Desktop-Clients für Windows, MacOS und Linux Betriebssysteme.
Zudem können Teilnehmer über ihren Webbrowser oder ihr Android oder iOS Mobilgerät an Meetings teilnehmen.
\autocite[Vgl.][]{M_Zoom.o.J.}
%(Vgl. \url{https://zoom.us/download#client_4meeting}, Abgerufen am 26.02.2021)
\\

\subsection{Blackboard Collaborate}
Blackboard Collaborate ist eine von \textit{Blackboard Inc.} bereitgestellte Onlinekonferenzlösung, die vor allem auf die Onlinelehre abzielt.
\autocite[Vgl.][]{M_Blackboard.o.J.}
%(Vgl. \url{https://www.blackboard.com/teaching-learning/collaboration-web-conferencing/blackboard-collaborate}, Abgerufen am 27.02.2021).
Blackboard Collaborate gibt an, besonders für die Bildung gemacht zu sein und auf die Bedürfnisse von Lehrenden und Lernenden angepasst zu sein.
Dazu bietet Blackboard Collaborate die Möglichkeit direkt aus dem Webbrowser heraus nutzbar zu sein.
Der Download eines besonderen Clients wird dabei nicht gebraucht.
Wie bei Zoom gibt es auch bei Blackboard Collaborate verschiedene Lizenzen, die mehr oder weniger Funktionen zur Verfügung stellen.
Es wird zwischen der Enterprise und der Departmentlizenz unterschieden.
Die Departmentlizenz bietet die Möglichkeit, dass bis zu 500 Teilnehmer an einer Konferenz teilnehmen können.
Zusätzlich können bis zu 500GB an Dateien gespeichert werden. Die Sitzungszeit ist auf 1000000 Minuten begrenzt.
Die Enterpriselizenz bietet wie die Departmentlizenz maximal 500 Teilnehmer die Möglichkeit an einer Session teilzunehmen.
Die Speicher- und Sitzungszeitbegrenzungen sind jedoch Variable.
Die Kosten für die Departmentlizenz betragen 9000 Dollar pro Jahr, für die Enterpriselinzenz, wird ein individueller Preis gezahlt, der von den gewünschten Funktionen abhängt.
\autocite[Vgl.][]{M_Blackboard.o.J.}
%(\url{Vgl. https://www.blackboard.com/teaching-learning/collaboration-web-conferencing/blackboard-collaborate}, Abgerufen am 27.02.2021).
\\
Das Tool kann auf verschiedene Weisen genutzt werden.
Es gibt die Möglichkeit wie in einem Klassenraum den Teilnehmern eine Anwendung zu teilen und Dinge zu präsentieren.
Zudem können Teilnehmer in kleineren Gruppen zusammenarbeiten.
\autocite[Vgl.][]{M_Blackboard.o.J.}
%(Vgl. \url{https://www.blackboard.com/teaching-learning/collaboration-web-conferencing/blackboard-collaborate}, Abgerufen am 27.02.2021).
\\

\subsection{BigBlueButton}
BigBlueButton ist wie Blackboard Collaborate ein Onlinekonferenzwerkzeug mit dem Fokus auf digitalem Lernen.
Genau wie Blackboard Collaborate ist BigBluebutton vor allem ein Web Konferenzsystem, dass in einem Webbrowser verwendet werden kann.
BigBlueButton ist ein Open Source Projekt.
\autocite[Vgl.][]{M_BigBlueButton.o.J.b}
%(Vgl. \url{https://bigbluebutton.org/}, Abgerufen am 27.02.2021).
\\
Bei Open Source Projekten ist der Programmcode für alle Menschen einsehbar.
Das bietet den Vorteil, dass Menschen aus der ganzen Welt an dem Projekt mitentwickeln können.
Durch Open Source können Fehler schneller entdeckt und behoben werden.
Zudem können Programme an individuelle Bedürfnisse angepasst werden.
\autocite[Vgl.][]{M_RedHat.o.J.}
%(Vgl. \url{https://www.redhat.com/de/topics/open-source/what-is-open-source}, Abgerufen am 27.02.2021).
\\
BigBlueButton bietet wie Blackboard Collaborate verschiedene Nutzungsmöglichkeiten.
Lehrende können beispielsweise ein digitales Whiteboard nutzen um Dinge zu visualisieren.
Die Zeichnungen können von den Sitzungsteilnehmern in Echtzeit verfolgt werden.
Um zwischenmenschliche Kommunikation zu ermöglichen können alle Teilnehmer per Videofeed an der Sitzung teilnehmen.
Es gibt dabei kein Limit, wie viele Teilnehmer ihre Kamera nutzen. Das einzige Limit ist die Internetbandbreite der Teilnehmer.
\autocite[Vgl.][]{M_BigBlueButton.o.J.}
%(Vgl. https://bigbluebutton.org/teachers/, Abgerufen am 27.02.2021).
\\
BigBlueButton bietet neben dem Teilen von Inhalten und halten von Videokonferenzen die Möglichkeit zu chatten, Emojis zu verwenden, Umfragen zu starten, sowie zusammen an Whiteboards zu arbeiten.
Zudem können Breakout Rooms erstellt werden, um so in kleineren Gruppen zusammenarbeiten zu können.
\autocite[Vgl.][]{M_BigBlueButton.o.J.}
%(Vgl. \url{https://bigbluebutton.org/teachers/}, Abgerufen am 27.02.2021).
\\

\subsection{Weitere Onlinekonferenzsysteme}
Neben den vorgestellten Lösungen für Onlinekonferenzen existiert eine Vielzahl weiterer Anwendungen.
Diese Werkzeuge sind dabei auf die unterschiedlichen Anwendungsfälle und Bedürfnisse der Benutzer angepasst.
Im Folgenden wird eine kleine Übersicht über weitere Onlinekonferenzsysteme und ihre Einsatzmöglichkeit gegeben.
Die Werkzeuge werden dabei nicht so detaliert vorgestellt, wie die in \autoref{sec:konferenzsysteme_vorgestellt} vorgestellten Lösungen.
\\
TODO: HIER EINE KLEINE LISTE BAUEN MIT WEITEREN SYSTEMEN...

\section{Vergleich der Konferenzsysteme}
Hier eine Tabelle rein und kurz was schreiben\\
TODO: ZU DEN EINZELNEN TOOLS AUF JEDEN FALL NOCH WAS SCHREIBEN